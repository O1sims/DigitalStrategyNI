\chapter{Digital skills}

In order to retain and develop highly productive, well-paid jobs in Northern Ireland, it is important that businesses have access to the skills they need, and which are important in order to develop and use new technologies to create innovation, new business models and growth.

Future growth and progress in Northern Ireland is also highly dependent on future generations becoming skilled IT users and understanding how to develop and analyse IT, so that individuals can not only participate in the digital society of the future, but also help to create it.

The companies' access to relevant skills is crucial in terms of which jobs can be established and retained in Northern Ireland. Over time, the labour market adjusts to the supply of and demand for skills, e.g. through the educational choices of young people, continuing education, etc., salary adjustments and attraction of a foreign labour, but also through discontinuing and offshoring those roles that cannot be filled.

If businesses have access to skills in fields such as technology, digital and science, they will be able to use the new technology to develop cutting-edge new products and create new business models.

Improved efficiency will thus strengthen the competitiveness of business. It will also create jobs in other fields.

If companies cannot meet their needs in areas such as digital skills, they are more likely to move jobs to countries where they can more easily find employees with the relevant skills.

\section{Access to digital and technical skills}

It is crucial for businesses to have access to employees with technical, digital and scientific skills (often known as STEM, which is an abbreviation of Science, Technology, Engineering and Mathematics). At the same time, we also need creative abilities and an understanding of business in order to develop, design and apply new technological and digital tools.

The proportion of the workforce in Northern Ireland with higher-level digital or technical education is relatively low compared to other European countries. This also applies to the younger generations, although the admission in general numbers are rising for higher educations within technical, digital and scientific subjects, such as a variety of engineering and IT educations.

We are starting to see the effects of the general increase in admission to higher education for STEM subjects over the past decade are starting to show. As such Northern Ireland is above the OECD average if we look at the number of completed STEM courses for the age group of 18--40 year olds. This is a positive development, which it is crucial to support in future.

On the other hand, the proportion of the workforce with vocational STEM education and training is declining.

\section{Recruitment challenges may slow down growth}

Northern Ireland is already one of the European countries with the most employees in IT and STEM-related jobs (European Labour Force Survey, 2013). IT and technology are expected to continue becoming an increasing part of most people's private and working lives. 

At the same time, we are in a situation in which companies face extensive challenges as regards recruitment in a variety of technical, digital and scientific fields. This may diminish the potential of a technological and digital transformation, and thereby the competitiveness of companies.

Some suggest that demand for a workforce with technical, digital and scientific skills will continue to rise, and there may be lacking around 10,000 engineers and natural science graduates by 2025.

The lack of digital skills can already present today. Approximately half of the businesses that tried to recruit IT specialists in 2016 report that their attempts were in vain (Report on Northern Ireland's Digital Growth, 2017). Many other countries are facing similar challenges.

At the same time, there is a lack for a range of STEM profiles, including an extensive nationwide lack of programmers and system developers, mechanical engineers and electricians in the Northern Irish labour market.

\section{Digital skills throughout the educational system}

There needs to be a greater focus on digital skills and technological understanding throughout the educational system. The Government has already taken several important steps to support its goal of strengthening the digital skills of individuals with reforms in both primary and secondary vocational education and training.

In the area of higher education, several new IT educations have been established in the recent years, while a more general focus has been applied to digital skills in many educations.

The Government wants to focus even more on how primary and lower secondary education, upper secondary education, higher education and adult and continuing education can equip the population with the right skills that match demand in trade and industry.

Primary and secondary education must focus on increasing technological understanding based on a basic democratic view that citizens should be able to participate in and influence processes and decisions that affect their lives. This also requires greater focus on technological understanding and digitalisation.

Higher education should train students to be among the best in the world so that Northern Ireland can be a catalyst for both new technology and new business models. There will also be a greater need to provide adult or continuing education for the individual in areas such as new technologies and digital tools.

With the October 2017 tripartite agreement, the Government and parties of the labour market focused on creating an Adult and Continuing Education system (ACE) that is both more flexible and meet the demand of the businesses. Finally, the ambition for the ACE system is to be better geared towards strengthening the digital skills of the entire workforce and adapt to the quickly fluctuating needs of the labour market.

This can for example be done by establishing a national strategic digitalisation initiative for the entire ACE area. A development fund will be allocated for this purpose over the next four years. We are also setting up a transformation fund to support mobility in the labour market, particularly as qualification requirements evolve in line with the technological progress among other things.

With the Strategy for Northern Ireland's Digital Growth, the Government will further strengthen the digital skills of Northern Irish citizens via a range of specific initiatives, some of which are aimed at employed and unemployed people, and some aimed at children and young people.

These initiatives build on a range of existing initiatives to help ensure that all Northern Irish citizens have the right tools to succeed in the digital transformation.

Among other things, this applies to the Government's initiatives on increased admission for higher education in STEM subjects, a national science strategy and a tripartite agreement on improved and more flexible adult, continuing and higher education in conjunction with labour market parties.
