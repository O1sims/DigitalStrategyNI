\setcounter{page}{1} \pagenumbering{arabic}
\chapter{Introduction} 

\section{Aim and motivation}
\label{sec:aimMotivation}

\section{Overview of digital strategies across Europe}

\subsection{Denmark}

In January 2018, the Danish Government published a strategy on advancing the Danish digital economy. The Digital Growth Strategy aims at placing Denmark in front when it comes to exploring the opportunities offered by digitisation and new technology. The strategy contains 38 concrete initiatives, which aim to secure and enhance Denmark’s position as an attractive digital hub by providing a boost to the Danish tech ecosystem and improve conditions for businesses to be able to utilise the full benefits of new technologies.

Among the initiatives are:
\begin{itemize}
\item A digital hub with a matchmaking platform to improve companies' access to talents and competences within emerging digital technologies.
\item A more agile regulation on companies' opportunities to test new business models.
\item Introducing computational thinking in elementary school.
\item A digital solution for companies to report cyber security incidents.
\end{itemize}

\subsection{Sweden}

In 2017 the Swedish Government a digital strategy focussing on sustainable digital transformation of the Swedish economy. The Agenda set the policy objective and established a Government advisory committee (the `Digitisation Commission'). The Commission, chaired by Sweden's Digital Champion, was tasked by Government to promote the digital agenda objective and engage stakeholders and to develop an action achieving the policy objective and present this to Government.

The overall policy objective is for Sweden to become the world leader in harnessing the opportunities of digital transformation. The approach recognises the need for leadership and the need to monitor progress. Priorities include: Regulation for digital age; digital security; digtial skills; data innovation; digital infrastructure; and importance of national leadership and continuous analysis of digital maturity and the need of measures.

\subsection{Finland}

Every year the Finnish Government releases its action plan, which encompasses many aspects of the Finnish economy, including digitisation and experimentation. Finland has made a productive leap in public services and the private sector by harnessing the potential offered by digitisation, specifically in dismantling unnecessary regulation and cutting red tape. The flexible regeneration of Finnish society is supported by a management culture based on trust, interaction and experimenting.

The Government has a number of objectives for the priority areas of digitisation:
\begin{itemize}
    \item Digitised public services. Public services are designed to be user-responsive and primarily digital. This is to be achieved by modernising practices and is crucial for achieving a productivity leap in public administration.
    \item Growth environment for digital business operations. The intention is to create a favourable operating environment for digital services and new business models. Specifically, big data and robotisation will be leveraged to create new businesses and practices.
    \item Streamlined legal provisions. The objective is to create enabling regulation, promoting deregulation and reducing administrative burden.
    \item Culture of experimenting. Experimentation will aim at innovative solutions, improvements in services, the promotion of individual initiative and entrepreneurship, and the strengthening of regional and local decisionmaking and cooperation.
    \item Better leadership and implementation. Government and central government management processes will be reconciled with the Government's strategy work. Knowledge-based management and implementation reaching across administrative branches will be strengthened.
\end{itemize}

\section{Current initiatives in Northern Ireland}

\subsection{Public sector}

\subsubsection{Department for the Economy (DfE)}

\paragraph{Academies}

\paragraph{Bring I.T. On}

\subsubsection{Department for Education}

\subsection{Private sector}

\subsubsection{Allstate}

\subsubsection{BT}

\subsubsection{Calayst Inc.}

\paragraph{Springboard.} Springboard is an in-business initiative that challenges, supports and expedites robust business models, including sales and marketing strategies, by leveraging the collective knowledge of a unique network of experienced entrepreneurs and business leaders in an open, nurturing environment. The programme is open to companies with the ambition and potential to grow revenue to between \pounds 10 million and \pounds 100 million.

The programme is established to \emph{challenge}, \emph{support} and \emph{expediate} new businesses and products by: (a) providing a network of experienced people that stress-test product ideas and help to refine the proposition; (b) provide unconditional support; and (c) helping to establish a robust business model quickly.

\paragraph{Generation Innovation.} .

\paragraph{Invest.} An annual competition that showcases and rewards local innovations and proof of concept ideas that have the greatest commercial potential.

\paragraph{Co-founders.} Co-founders is a 10 week programme that brings together individuals and teams with experience and ideas to develop new business products and services.

\paragraph{CEOs Connect.} CEOs Connect is a peer-to-peer network supporting CEOs of high growth potential innovation companies in Northern Ireland as they scale and grow. The network allows members to develop relationships in confidence with other CEOs of innovation companies, share intelligence and networks, and grow and scale their companies and deal with any challenges they have.

\subsubsection{Deloitte}

\subsubsection{PricewaterhouseCoopers (PwC)}