\setcounter{page}{1} 
\pagenumbering{arabic}

\chapter{Introduction} 

\section{Aim and motivation} \label{sec:aimMotivation}


\section{Overview of Northern Ireland's Digital Economy}

\subsection{What is the Digital Economy?}

To analyse the Digital Economy and the role it plays in Northern Ireland, the most significant issue is with regards its definition. \citet{Mesenbourg2001} defines the \emph{Digital Economy} in terms of the following components:
\begin{itemize}
    \item e-commerce / e-business (i.e. the trading of goods or services over digital networks such as the Internet)
    \item supporting physical and digital infrastructure
\end{itemize}

In developing a strategy to support this `sector' of the economy we also condier institutional elements that either support or draw from the Digital Economy in Northern Ireland, namely organisations established to support the: 
\begin{itemize}
    \item ideation, incubation, acceleration and scaling of technology firms
    \item supply digital skills to the economy
    \item digitisation of public services
    \item development of technology clusters
    \item provision and sustainability of foreign direct investment (FDI)
\end{itemize}

The integrated nature of the Digital Economy was highlighted by Nesta. They noted that businesses may not be described as digital but employ staff in a role which would be considered as digital, and vice versa \citep{Spilsbury2016}. Their report showed that only 47\% of those working in ICT-related jobs, in 2015, were employed in an ICT sector company, meaning that a large proportion of what may be considered to be the Digital Economy is not captured using analysis based on traditional industrial classifications.

\subsection{The link between the Digital Economy and the Digital Sector?}

To measure the Digital Economy, we need to isolate those businesses which comprise the broad concept of the Digital Economy. The Organisation for Economic Co-operation and Development (OECD) defines the ICT sector (an approximation to the Digital Economy) as, ``a combination of manufacturing and service industries that capture, transmit and display data and information electronically''. The OECD and UK Government considers the following businesses to encompass this concept as defined by the Standard Industrial Classification (SIC) codes given in Table \ref{table:SICcodes}.

\begin{table}[]
\label{table:SICcodes}
\centering
\begin{tabular}{cl}
\hline
SIC code    & Description                                              \\
\hline
26.1        & Manufacture of electronic components and boards          \\
26.2        & Manufacture of computers and peripheral equipment        \\
26.3        & Manufacture of communication equipment                   \\
26.4        & Manufacture of consumer electronics                      \\
26.8        & Manufacture of magnetic and optical media                \\
46.5        & Wholesale of information and communication equipment     \\
58.1        & Publishing activities                                    \\
58.2        & Software publishing                                      \\
59.1--59.2  & Film, video \& television programme production           \\
60.1        & Radio broadcasting                                       \\
60.2        & Television programming and broadcasting activities       \\
61.1--61.9  & Telecommunications                                       \\
62          & Computer programming, consultancy and related activities \\
63.1--63.9  & Information service activities                           \\
95.1        & Repair of computers and communication equipment          \\
\hline
\end{tabular}
\caption{Standard Industrial Classification (SIC) codes for firms within Digital Economy}
\end{table}

\subsection{How does the Digital Sector contribute to the economy?}

According to The 2018 Northern Ireland Knowledge Economy Report \citep{KnowledgeEconomy2019}, which provides a representation of the Digital Economy, the local real Gross Value Added (GVA) of the Digital Economy generated in 2016 was \pounds 2.2bn (out of a \pounds 37bn regional economy). This accounts for 5.9 \% of Northern Ireland's total GVA. Northern Ireland is significantly below the UK GVA average of around 10\%, and is ranked tenth among the regions. The compound annual growth rate of the knowledge economy in Northern Ireland is 6.3\%, suggesting that Northern Ireland has the fourth fastest growing knowledge economy in the UK. The rate of growth is good, although it comes from a small base.

Approximately 40,250 workers are directly employed by the Digital Economy and an extra 30,000 workers are employed indirectly and as an induced effect of the digital economy. As a consequence, the digital sector has one of the largest Type II multipliers in the Northern Irish economy \footnote[1]{For more information on economic multiplers please visit:\\ \href{https://www2.gov.scot/Topics/Statistics/Browse/Economy/Input-Output/Mulitipliers}{https://www2.gov.scot/Topics/Statistics/Browse/Economy/Input-Output/Mulitipliers}} at approximately 1.74.

\begin{center}
    INSERT CHARTS FOR KNOWLEDGE ECONOMY INDEX \& JOBS GENERATED
\end{center}

In 2018 Northern Ireland remained the second-fastest-growing knowledge economy in the UK. The survey of activity was broadly positive, with improvement reported in almost two-thirds of indicators.  



\section{Digital strategies across Europe}

\subsection{Denmark}

In January 2018, the Danish Government published a strategy on advancing the Danish digital economy. The Digital Growth Strategy aims at placing Denmark in front when it comes to exploring the opportunities offered by digitisation and new technology. The strategy contains 38 concrete initiatives, which aim to secure and enhance Denmark's position as an attractive digital hub by providing a boost to the Danish tech ecosystem and improve conditions for businesses to be able to utilise the full benefits of new technologies.

Among the initiatives are:
\begin{itemize}
\item A digital hub with a matchmaking platform to improve companies' access to talents and competences within emerging digital technologies.
\item A more agile regulation on companies' opportunities to test new business models.
\item Introducing computational thinking in elementary school.
\item A digital solution for companies to report cyber security incidents.
\end{itemize}

\subsection{Sweden}

In 2017 the Swedish Government a digital strategy focussing on sustainable digital transformation of the Swedish economy. The Agenda set the policy objective and established a Government advisory committee (the `Digitisation Commission'). The Commission, chaired by Sweden's Digital Champion, was tasked by Government to promote the digital agenda objective and engage stakeholders and to develop an action achieving the policy objective and present this to Government.

The overall policy objective is for Sweden to become the world leader in harnessing the opportunities of digital transformation. The approach recognises the need for leadership and the need to monitor progress. Priorities include: Regulation for digital age; digital security; digtial skills; data innovation; digital infrastructure; and importance of national leadership and continuous analysis of digital maturity and the need of measures.

\subsection{Finland}

Every year the Finnish Government releases its action plan, which encompasses many aspects of the Finnish economy, including digitisation and experimentation. Finland has made a productive leap in public services and the private sector by harnessing the potential offered by digitisation, specifically in dismantling unnecessary regulation and cutting red tape. The flexible regeneration of Finnish society is supported by a management culture based on trust, interaction and experimenting.

The Government has a number of objectives for the priority areas of digitisation:
\begin{itemize}
    \item Digitised public services. Public services are designed to be user-responsive and primarily digital. This is to be achieved by modernising practices and is crucial for achieving a productivity leap in public administration.
    \item Growth environment for digital business operations. The intention is to create a favourable operating environment for digital services and new business models. Specifically, big data and robotisation will be leveraged to create new businesses and practices.
    \item Streamlined legal provisions. The objective is to create enabling regulation, promoting deregulation and reducing administrative burden.
    \item Culture of experimenting. Experimentation will aim at innovative solutions, improvements in services, the promotion of individual initiative and entrepreneurship, and the strengthening of regional and local decisionmaking and cooperation.
    \item Better leadership and implementation. Government and central government management processes will be reconciled with the Government's strategy work. Knowledge-based management and implementation reaching across administrative branches will be strengthened.
\end{itemize}

\section{Current initiatives in Northern Ireland}

\subsection{Public sector}

\subsubsection{Department for the Economy (DfE)}

\paragraph{Academies}

\paragraph{Bring I.T. On}

\subsubsection{Business in the Community}

\paragraph{Time to Code.} A volunteering programme run by Business in the Community in partnership with Code Club. It aims to help children at Key Stage 2 Level gain IT and coding skills, build heir confidence, and develop their team working and problem solving abilities. The programme is supported by Belfast Harbour in the Greater Belfast Area, and BT in the North West.

\subsubsection{Department of Finance (DoF)}

\paragraph{Digital Transformation Strategy 2017--2021.} The strategy comes on the back of the Programme for Government (PfG) which includes a commitent to increase the use of online channels, with 70\% of all citizen transactions with Government online by 2019. Overall, the strategy sets out a vision for transforming how Government works, facilitating Governments embrace of digital in everyday public services to deliver better outcomes. As a consequence, Government operation are to be digitally transformed to deliver effective and efficient public services.

\paragraph{Go ON NI digital initiative.} Go ON NI is a government initiative which introduces digital technology to people who aren't familiar with going online and supports beginners who want to improve their online skills.

\paragraph{Open data NI.} The open data team within Digital Transformation Service has developed a NI Open Data Strategy which supports making non-personal government-held electronic data sets freely available through license for use by anyone, without restriction. The Northern Ireland Executive has committed to publication of Public Sector data as Open Data, and to that end, the `Open Data Strategy for Northern Ireland 2015–18' was endorsed and published in February 2015.

\subsubsection{Department for Education}

\subsubsection{Belfast City Council}

\paragraph{Digital Futures.} The programme, organised by Digital DNA, assists 40 teenagers, aged 16--18 years old, in realising their potential and the opportunities available to them within the digital and technology sectors. 

\subsection{Private sector}

\subsubsection{Allstate}

\subsubsection{BT}

\paragraph{Barefoot.} Funded and run by BT in partnership with Computing at School (CAS), the project is designed to help primary school teachers across the UK prepare for the changing computing curriculum. The content is developed by teachers for teachers. The Royal Society has acknowledged that Barefoot is the primary teacher's resource of choice for computational thinking in the classroom. The programme provides classroom and teacher resources as well as Barefoot workshops.

\paragraph{FabLab.} Digital Fabrication Laboratories provide the environment, skills, advanced materials and technology to make things cheaply and quickly anywhere in the world, and to make this available on a local basis to entrepreneurs, individuals, students, artists, and small businesses.

\subsubsection{Calayst Inc.}

\paragraph{Springboard.} Springboard is an in-business initiative that challenges, supports and expedites robust business models, including sales and marketing strategies, by leveraging the collective knowledge of a unique network of experienced entrepreneurs and business leaders in an open, nurturing environment. The programme is open to companies with the ambition and potential to grow revenue to between \pounds 10 million and \pounds 100 million.

The programme is established to \emph{challenge}, \emph{support} and \emph{expediate} new businesses and products by: (a) providing a network of experienced people that stress-test product ideas and help to refine the proposition; (b) provide unconditional support; and (c) helping to establish a robust business model quickly.

\paragraph{Generation Innovation.} An initiative established to prepare young poeple to thrive as future leaders in enterprise and the changing workplace. The initiative includes:
\begin{itemize}
    \item Night of Ambition: An annual event providing an opportunity for teachers and parents to get up to speed on emerging trends in technology and knowledge-based careers in NI.
    \item Day of Ambition: Workshops that are driven to educate young people who are curious to achieve their potential in an ever-changing world.
    \item Re-Imagining Work Experience: Unique work experience programme designed in collaboration with \emph{Deloitte} and delivered with 10 of NI's most innovative companies.
\end{itemize}

\paragraph{Invent.} An annual competition that showcases and rewards local innovations and proof of concept ideas that have the greatest commercial potential.

\paragraph{Co-founders.} Co-founders is a 10 week programme that brings together individuals and teams with experience and ideas to develop new business products and services.

\paragraph{CEOs Connect.} CEOs Connect is a peer-to-peer network supporting CEOs of high growth potential innovation companies in Northern Ireland as they scale and grow. The network allows members to develop relationships in confidence with other CEOs of innovation companies, share intelligence and networks, and grow and scale their companies and deal with any challenges they have.

\paragraph{4IRC.} The 4th Industrial Revolution Challenge is an annual series of monthly headline discussions held across Northern Ireland encouraging participants to learn together how the following sectors: health, finance, energy, transport, government and food, will be disrupted by accelerations in the following platform technologies: IoT, Analytics, AI, VR, 3D printing, robotics, genomics, and quantum computing. 

\subsubsection{Deloitte}

\subsubsection{PricewaterhouseCoopers (PwC)}

\paragraph{Hive Tech Academy.} An initiative that intends to develop digital skill sets in young people. The team from PwC go to schools across Northern Ireland to teach pupils insights into design thinking, cyber security and data analytics.

\subsubsection{Belfast Metropolitan College}

\paragraph{Connected.} Works with businesses in areas such as Emerging Technology and New Product Development. Belfast Met focusses on work in the following areas: cyber security, data visualisation, advanced materials, and life \& health science.

\paragraph{CoderDojo.} A movement orientated around running free not-for-profit coding clubs which teach young people to learn how to write computer code, develop websites, apps, programs and games.

\paragraph{Advanced composites.} 

\paragraph{Belfast IT Girls.} An annual event that brings 

\subsubsection{Farset Labs}

\paragraph{Code Co-Op.} A bi-weekly club for adults to come together and work on fun projects to upskill themselves at Farset Labs.

\subsubsection{Kainos}

\paragraph{Code Camp.} Initiative that provides an opportunity for STEM students to learn how to build apps and websites from scratch, take part in interactive sessions with experts and hear talks on a range to cyber-related topics.

\paragraph{AI Camp.} A 2-week long camp, full of learning, workshops and practical sessions to give students a good introduction to artificial intelligence and machine learning, and how it can be used to benefit our everyday lives. Members of the Kainos team provide formal training over the 2 weeks and provide the fundamental knowledge and skills needed.

\paragraph{AI NI.} A regular meet-up for the Arificial Intelligence (AI) community of Northern Ireland. The meetups include companies, students and academia. Each event comprises presentations, workshops and panel discussions designed to educate and stimulate discussion.

\subsubsection{Instil}

\paragraph{Bash!} A regular meet-up for developers organised by developers. The focus is on current and upcoming trends in software.

\subsection{Tech Education Network (TEN)}

A network based in Northern Ireland engaging in youth outreach from across the tech and design community.

\subsubsection{Meet-ups}

\paragraph{Charged.} A team dedicated to creating a friendly, inclusive environment for attendees to make new connections, work with the latest technology and learn something new. They host events aimed at helping people get the most out of their studies, career or hobby and we want to create a community that they want to contribute to and get involved with.

\subsubsection{Hubs and co-working spaces}

\paragraph{Ormeau Baths.} A co-working space and tech community in Belfast. Entrepreneurial campus to help indigenous tech start-ups gain access to new networks, funding and expertise and to give them the tools to access and grow in global markets.

\subsubsection{Misc.}

\paragraph{NodeSchool.} Hosted by Farset Labs, and supported by White Hat in Belfast, NodeSchool is an open source initiative run by volunteers with two goals: to create high quality programming curriculum and to host community learning events.

\paragraph{Code First: Girls.} An initiative that trains women in IT skills and helps companies to develop more female-friendly recruitment policies. The organisation promotes gender diversity and female participation in the technology sector by offering free and paid training and courses for students and professional women. It also supports businesses to train staff and develop talent management policies.

\paragraph{Django Girls.} Free programming workshops for women from the age of 18 and over, focussed primarily on backend Python development.

\paragraph{Code Club NI.} A network of volunteers and educators who run free coding clubs for children aged 9--13.

\paragraph{Belfast Linus User Group (BLUG).} Users, hobbyists, coders and administrators with an interest in Linux, open source technology and the FLOSS movement. A mixture of social and technical sides of Linux.

\paragraph{Belfast MongoDB Meetup.} A group dedicated to increasing knowledge and awareness about MongoDB, an Open Source, document-oriented, NoSQL database.

\paragraph{The Belfast .NET Developer Guild.} A group interested in developing code and applications in the .NET stack, covering coding best practice, new developments and complimentary tooling. Solely focussed on .NET development.

\paragraph{PyBelfast.} A group for anyone interested in learning more about the Python programming language. We aim to hold a meetup every other month on a variety of topics, with a focus on Python related technologies.

\paragraph{Belfast Puppet User Group.} 