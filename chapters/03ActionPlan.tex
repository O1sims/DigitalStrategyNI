\chapter{Digital skills action plan}

\section{Defining the skills pipeline}

\section{Skills provision}
\label{sec:skillsProvision}

\subsection{Understanding the demand for ICT professionals}

Understanding the demand for ICT professionals is vital, if we are to ensure that there is a
pipeline in place to deliver an appropriate number of quality skilled people to work in the
industry. In order to provide this understanding of demand in the ICT sector, four main
sources of research have been used.

\paragraph{Ulster University Skills Barometer}



\subsubsection{Actions}

\begin{itemize}
    \item The CBI will use the current Skills Barometer and labour market information to update forecasting reports to identify specific skills gaps faced by each sub-sector (software development, infrastructure management and applications management) and likely future skills gaps. Consideration may need to be given with respect to the current and forseen FDI strategy.
    \item The CBI will seek support from Invest NI to gather information on the demand side to better understand the range of skills demands in the sector.
    \item The CBI will seek support from Invest NI to assess the future skills demands from target FDI companies.
    \item The CBI will provision a regional survey to its members and other related companies and organisations their skills demands.
\end{itemize}

\subsection{Understand the current and projected supply of appropriately qualified people relevant to employer need}

\subsubsection{Actions}

\begin{itemize}
    \item CCEA and Department for Education to provide spreadsheets on the uptake and grade success of ICT-related fields. 
    \item Universities and colleges will provide figures regarding predicted outflow from their courses to develop a clear understanding of the provision from our institutions in order to inform actions.
    \item Universities and colleges will provide data regarding where computer science and engineering students work (company, industry, country) after graduating.
    \item Continual monitoring of the dynamics between the sector’s demand for graduates and the numbers supplied through the education and training systems.
    \item Industry to release figures on 
\end{itemize}

\paragraph{Skills pipeline areas.} 

\subsection{Address critical skills shortages}

\subsubsection{Actions}

\begin{itemize}
    \item The Department for the Economy and ICT employers will meet regularly to develop the Academy approach to address known skills shortages that can be used in the industry as a whole.
    \item Generalist and specialist courses to be discussed and arranged. Generalist courses to provide skills that are intended to benefit the workforce as a whole and specialist in certain technologies.
    \item The generalist and specialist courses - which has been developed by the Department for the Economy and the further education sector - will be validated by the industry, to align the curriculum more closely with the needs of employers.
    \item Based on the agreed identified skills gaps, the CBI will identify the existing relevant provision which meets the agreed skills needs and ways in which uptake can be increased.
    \item The CBI to understand how syllabuses in colleges and Universities can be tweaked to facilitate the supply of relevant skills.
\end{itemize}

\subsubsection{Encourage the take up of GCSE and A-Level students for ICT and computer studies}

\subsubsection{Actions}

\begin{itemize}
    \item The Department of Education, via CCEA, will continue to examine the current provision and range of school computing / ICT qualifications to ensure it meets the needs of ICT courses in universities.
    \item Ensure Careers teachers in schools and Careers Advisers have access to up to date labour market information to highlight the career paths, opportunities and benefits associated with the ICT Sector to school children and their influencers.
\end{itemize}

\subsubsection{Incentivise the take up of relevant courses at college and university}

\subsubsection{Actions}

\begin{itemize}
    \item Universities will consider the use of bursaries / scholarships to encourage more people to study Computer Science, Engineering or relevant courses.
    \item Industry, in a concerted effort (through, say, the establishment of a new programme), will support schools with the provision of ‘guru lecturers’ and through assisting with the continued professional development of teachers.
    \item Universities and further education colleges will ensure a significant reduction of the high attrition rate from ICT and Computer Science courses.
\end{itemize}

\subsubsection{Address attrition rates in college and university}

\subsubsection{Actions}

\begin{itemize}
    \item Universities and further education colleges to review curriculum and ensure a significant reduction of the high attrition rate from ICT and Computer Science courses.
\end{itemize}

\subsection{Explore alternative routes into the ICT Sector}

\subsubsection{Actions}

\begin{itemize}
    \item ICT employers to inform a best practice model to recruit and train apprentices for work within the ICT sector.
    \item Invest NI and the Department for the Economy to investigate the possibility of establishing an assistance programme for SMEs that encourages and supports them in offering bursaries, scholarships, internships, apprenticeships and / or placements to students on approved courses.
\end{itemize}

\subsection{Attract students that leave NI for University education}

\subsubsection{Actions}

\begin{itemize}
    \item Industry to target students that leave and offer high quality winter and summer placements to attract them back after edcation.
\end{itemize}

\section{Sector attractiveness}
\label{sec:sectorAttractiveness}

Raising the profile of the ICT sector remains a major challenge for the industry. It is important that young people, the unemployed and those currently in employment are aware of the opportunities for employment and career progression in the ICT sector.

\subsection{Encouraging more people to choose the relevant subjects required to work in the sector}

\subsubsection{Actions}

\begin{itemize}
    \item Conduct an evaluation of Belfast MET’s 'Bring IT On' programme. Need to consider the effectiveness of the communication of key messages to the target audience(s) and their impact. The existing audience will be expanded to include younger children, parents and teachers.
    \item Discuss with industry how .
\end{itemize}

\subsection{Promote career opportunities to primary and second level students}

\begin{itemize}
    \item Highlight and promote STEM career opportunities and career pathways open to students, particularly those in ICT fields.
    \item Carry out a national survey of undergraduate students to identify why they did or did not choose STEM courses after school.
    \item Develop and manage a database of trained STEM volunteers including people with a variety of ICT backgrounds, to engage with schools and teachers nationwide and give career talks.
    \item Deliver career pathways information in an easily accessible manner online.
    \item Deliver a series of career profiles highlighting female role models in STEM, including those with ICT backgrounds.
    \item Support and promote competitions aimed at post-primary students encouraging them to develop STEM skills using innovate communications methods.
\end{itemize}

\subsection{Enhance higher education awarenessraising measures}

\begin{itemize}
    \item HEA to provide support to institutions in delivering Summer Computing Camps to encourage second-level students to consider ICT careers.
    \item Higher education institutions to continue to support Coder Dojo in provision of space, administrative supports and mentoring.
    \item Enhanced focus on attracting female participants, particularly in calls for targeted provision.
\end{itemize}

\subsection{Include ICT in Continuous Professional Development for Teachers}

\begin{itemize}
    \item Inclusion of ICT in all continuing professional development for teachers for example a new course has been developed. Also, active Learning in Literacy and Numeracy using Tablet Devices.
\end{itemize}

\subsection{Run an annual advanced ICT talent management and retention seminar to share best practice among companies in upskilling and HR talent management.}

\begin{itemize}
    \item The CBI to organise and run an annual advanced ICT talent management and retention seminar.
\end{itemize}